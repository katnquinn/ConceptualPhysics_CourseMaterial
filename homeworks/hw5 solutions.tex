\documentclass[addpoints,12pt]{exam}
%\documentclass[12pt]{article}
\usepackage[letterpaper, margin=0.75in]{geometry}
\usepackage{graphicx}
\usepackage{enumitem}
\usepackage{booktabs}
\usepackage{tabularx}
\usepackage{color}

\begin{document}
\footer{}{Page \thepage\ of \numpages}{}


\begin{center}
\includegraphics[width=10cm]{../images/logo.png}
\end{center}

\begin{center}
\noindent{\LARGE Conceptual Physics \\ Homework Packet 5\\ Solutions \\}
\end{center}

\clearpage

\begin{flushright}
Score: \hspace{0.2in} / \numpoints ~ points
\end{flushright}

\begin{questions}
\question[2] Does motion affect the rate of a clock as measured by an observer moving with it? Does motion affect how an observer moving relative to a clock
measures its rate?

From \textit{College Physics}, Chapter 28 Question 4
\begin{TheSolution}
\begin{itemize}
	\item The observed rate of a moving clock is \textit{unaffected} to an observer moving with it (they would measure \textit{proper time})
	\item An observer moving relative to a clock would perceive \textit{dilated} time (slower ticking)
\end{itemize}
\end{TheSolution}

\question[2] To whom does the elapsed time for a process seem to be longer, an observer moving relative to the process or an observer moving with the
process? Which observer measures proper time?

From \textit{College Physics}, Chapter 28 Question 5
\begin{TheSolution}
Longer: To an observer moving relative to a process. Proper time is measured by an observer moving with the process.
\end{TheSolution}

\question[2] A set of twins works in the Sears Tower, a very tall office building in Chicago. One works on the top floor and one works in the basement. Considering general relativity, which twin will age more slowly?
\begin{TheSolution}
The one in the stronger gravitational field will age more slowly: therefore the twin working in the basement will age more slowly than the twin working on the top floor.
\end{TheSolution}

\question[2] Does your body's mass cause spacetime to curve around it? If not, why not? If so, and if light follows curved paths through spacetime, why can't you see directly behind a person standing in your line of sight?
\begin{TheSolution}
Yes, your body causes spacetime to bend around it, however because people are so light (compared to a galaxy, which is very massive) we bend spacetime much less, and so light is only slightly deflected when it comes near us. This effect is so small we can't notice it. The warping isn't enough to cause light to bend the light coming from someone standing in your line of sight to bend around another person in between you: and so you don't see them.
\end{TheSolution}

\question[10]Signals are sent to and from satellites in orbit around Earth. If someone on Earth sends a radio signal up to one of these satellites,
	\begin{parts}
		\part Will the satellite perceive the signal to be red-shifted, blue-shifted, or the same as an observer on Earth?
			\begin{TheSolution}
			\textbf{Redshifted.} Light goes from a large to small gravitational field when it rises from the surface of the Earth to a satellite, and so is redshifted.
			\end{TheSolution}
		\part Will the satellite perceive the frequency to be larger, smaller or the same as an observer on Earth?
			\begin{TheSolution}
			\textbf{Smaller.} When light is redshifted, frequency decreases.
			\end{TheSolution}
		\part Will the satellite perceive the wavelength to be longer, shorter or the same as an observer on Earth?
			\begin{TheSolution}
			\textbf{Longer.} When light is redshifted, wavelength increases.
			\end{TheSolution}
		\part Will the satellite perceive the energy of the incoming radio waves to be greater, smaller or the same as an observer on Earth?
			\begin{TheSolution}
			\textbf{Smaller.} As light is redshifted, it loses energy.
			\end{TheSolution}
		\part Will the satellite perceive the radio waves to be travelling faster, slower or at the same speed as observer on Earth?
			\begin{TheSolution}
			\textbf{The Same.} All observers measure light to be traveling at the same speed.
			\end{TheSolution}
	\end{parts}

\question[8] A species of space-faring aliens are travelling in a spaceship at 1/3 the speed of light relative to a small uninhabited planet with no atmosphere. Their ship is equipped with two identical cannons that blast explosive devices straight down, creating craters to mine for various materials. The cannons on the ship are separated by a distance $L$, and the ship itself has a height $H$. As they pass by a planet, a distance $D$ above the surface, the aliens simultaneously launch 2 explosive devices (in their reference frame), creating 2 large craters A and B. The diagram bellow (not drawn to scale) illustrates this situation.
\begin{center}
\input{../images/AlienSpaceship.pdf_tex}
\end{center}
\begin{parts}
	\part To an alien standing on the planet, how far apart are the two cannons?
		\begin{choices}
			\choice A distance greater than $L$.
			\choice A distance $L$.
			\choice A distance less than $L$.
			\choice There is insufficient information to answer this question.
		\end{choices}
		\begin{TheSolution}
		\textbf{C.} From the perspective of an observer on the planet, the ship is moving from right to left, and so its length will be contracted.
		\end{TheSolution}
	\part What would an alien standing on the planet measure the height of the ship to be?
		\begin{choices}
			\choice Greater than $H$.
			\choice Exactly $H$.
			\choice Less than $H$.
			\choice There is insufficient information to answer this question.
		\end{choices}
		\begin{TheSolution}
		\textbf{B.} Since the height is perpendicular to the direction of motion (up-down is perpendicular from right-left) and so is not contracted.
		\end{TheSolution}
	\part To an alien on the planet, how high up is the ship?
		\begin{choices}
			\choice A distance greater than $D$.
			\choice A distance equal to $D$.
			\choice A distance less than $D$.
			\choice There is insufficient information to answer this question.
		\end{choices}
		\begin{TheSolution}
		\textbf{B.} The ship's altitude is perpendicular to the direction of motion, and so does not appear contracted to an observer on the planet.
		\end{TheSolution}
	\part To an alien on the ship, how high up are they flying?
		\begin{choices}
			\choice A distance greater than $D$.
			\choice A distance equal to $D$.
			\choice A distance less than $D$.
			\choice There is insufficient information to answer this question.
		\end{choices}
		\begin{TheSolution}
		\textbf{D.} same reasoning as above
		\end{TheSolution}
	\part To an alien on the ship, what is the distance separating the craters A and B?
		\begin{choices}
			\choice Greater than $L$.
			\choice Exactly $L$.
			\choice Less than $L$.
			\choice There is insufficient information to answer this question.
		\end{choices}
		\begin{TheSolution}
		\textbf{B.} From the perspective of the aliens on the ship, the distance between the two craters is the same as the distance between the cannons: \textit{L}
		\end{TheSolution}
	\part To an alien on the planet, what is the distance separating the craters A and B?
		\begin{choices}
			\choice Greater than $L$.
			\choice Exactly $L$.
			\choice Less than $L$.
			\choice There is insufficient information to answer this question.
		\end{choices}
		\begin{TheSolution}
		\textbf{A.} To observers on the ship, the planet is length contracted, and the distance between the craters is \textit{L.} For observers on the planet, the length is not contracted, and so it would be greater than \textit{L.}
		\end{TheSolution}
	\part To an alien on the ship, in what order are craters A and B created?
		\begin{choices}
			\choice Crater A is created first, then crater B.
			\choice Crater B is created first, then crater A.
			\choice They are created simultaneously.
			\choice There is insufficient information to answer this question.
		\end{choices}
		\begin{TheSolution}
		\textbf{C.} Since, from the perspective of an observer on the ship the cannons are fired simultaneously.
		\end{TheSolution}
	\part To an alien on the planet, in what order are craters A and B created?
		\begin{choices}
			\choice Crater A is created first, then crater B.
			\choice Crater B is created first, then crater A.
			\choice They are created simultaneously.
			\choice There is insufficient information to answer this question.
		\end{choices}
		\begin{TheSolution}
		\textbf{B.} Since the distance between the two cannons, from the perspective of someone on the planet is less than \textit{L} and the distance separating the craters is greater than \textit{L} the two cannot have been formed simultaneously. Drawing the trajectories, the crater on the right (B) must be formed before the one on the left (A) since the ship is moving from right to left (from the perspective of an observer on the planet).
		\end{TheSolution}
\end{parts}

\end{questions}








\end{document}