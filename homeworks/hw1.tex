\documentclass[addpoints,12pt]{exam}
%\documentclass[12pt]{article}
\usepackage[letterpaper, margin=0.75in]{geometry}
\usepackage{graphicx}
\usepackage{enumitem}
\usepackage{booktabs}
\usepackage{tabularx}

\begin{document}
\footer{}{Page \thepage\ of \numpages}{}

\begin{flushright}
\makebox[0.5\textwidth]{\large Name:\enspace\hrulefill}
\vspace{0.2in}

\makebox[0.5\textwidth]{\large Date:\enspace\hrulefill}
\end{flushright}

\begin{center}
\includegraphics[width=10cm]{../images/logo.png}
\end{center}

\begin{center}
\noindent{\LARGE Conceptual Physics \\ Homework Packet 1}
\end{center}

\noindent\begin{large}\textbf{Due: Feb 16, 2018}\end{large}
\vspace{0.2in}

Answer the questions in the spaces provided on the question sheets. If you run out of room for an answer, continue on the back of the page. If questions are taken from one of the textbooks, it will be indicated. A large portion of your grade will be calculated based on \textit{how} you obtained an answer, so please \textbf{show your work} (including all diagrams and drawings if relevant).

If you prefer working on loseleaf paper, or have a large portion of your work on loseleaf, please be sure to hand that in along with this homework packet.

The content in this homework relates to material we covered in class 1 (philosophy of science) and class 2 (units, including scientific notation and speed). The related readings are:
\begin{enumerate}
\item Class notes, and paper in course reader titled ``Is Science an Ideology"?
\item \textit{Light and Matter}, Chapter 0 (Sections 5, 8 and 9)
\end{enumerate}

The following table of metric prefixes may be useful:
\vspace{0.2in}

\begin{tabularx}{\textwidth}{ X X X X X X }
	kilo & (none) & centi & mili & micro & nano \\
	k & (none) & c & m & $\mu$ & n \\
	$10^3$ & $10^0$ & $10^{-2}$ & $10^{-3}$ & $10^{-6}$ & $10^{-9}$ \\
	1000 & 1 & 0.01 & 0.001 & 0.000001 & 0.000000001 
\end{tabularx}
 
\clearpage

\begin{flushright}
Score: \hspace{0.2in} / \numpoints ~ points
\end{flushright}

\begin{questions}
\question[5]
Acupuncture is a traditional medical technique of Asian origin in which small needles are inserted in the patient’s body to relieve pain. Many doctors trained in the west consider acupuncture unworthy of experimental study because if it had therapeutic effects, such effects could not be explained by their theories of the nervous system. Who is being more scientific, the western or eastern practitioners?

From: \textit{Light and Matter}, Ch.0, Discussion Question A.
\fillwithlines{2.5in}

\question[5]
A child asks why things fall down, and an adult answers ``because of gravity." The ancient Greek philosopher Aristotle explained that rocks fell because it was their nature to seek out their natural place, in contact with the earth. Are these explanations scientific? 

From: \textit{Light and Matter}, Ch.0, Discussion Question C.
\fillwithlines{2.5in}

\question[4]
Compute the following things. If they don't make sense because of units, say so.

From: \textit{Light and Matter}, Ch.0, Q. 2.
\begin{parts}
 \part 3 cm + 5 cm
	\vspace{0.3in}
 \part 1.1 m + 22 cm
 	\vspace{0.3in}
 \part 120 miles + 2.0 hours
 	\vspace{0.3in}
 \part 120 miles / 2.0 hours
	\vspace{0.3in}
\end{parts}

\question[2]
Convert 134 mg to units of kg, writing your answer in scientific notation.

From: \textit{Light and Matter}, Ch.0, Q. 6
\vspace{0.5in}

\question[5]
Express each of the following quantities in micrograms:

From: \textit{Light and Matter}, Ch.0, Q. 5:
\begin{parts}
 \part 10 mg
	\vspace{0.3in}
 \part $10^4$ g
	\vspace{0.3in}
 \part 10 kg
	\vspace{0.3in}
 \part $100\times 10^3$ g
	\vspace{0.3in}
 \part 1000 ng
	\vspace{0.3in}
\end{parts}

\question[3]
Your backyard has brick walls on both ends. You measure a distance of 23.4 m from the inside of one wall to the inside of the other. Each wall is 29.4 cm thick. How far is it from the outside
of one wall to the outside of the other?

From: \textit{Light and Matter}, Ch. 0, Q. 3
\vspace{3in}



\end{questions}



\end{document}