\documentclass[12pt]{article}
\usepackage[letterpaper, margin=0.75in]{geometry}
\usepackage{tabularx}
\usepackage{graphicx}
\usepackage{titling}


\pretitle{\begin{center}\LARGE\includegraphics[width=10cm]{../images/logo.png}\\[\bigskipamount]}
\posttitle{\end{center}}

\begin{document}

\title{Conceptual Physics Syllabus}

\author{Instructor: Ms. Quinn} 
	
\maketitle

\noindent \textbf{Course Description}

This course will serve to introduce the concepts that form the basis of our understanding of physical theories. We will cover a wide range of topics, placing them in the historical and cultural context in which they arose, from introductory mechanics involving familiar notions like distance, speed, force, etc. to more exotic ones from special and general relativity, and quantum mechanics. This will be primarily a concept based course, involving little mathematics beyond basic algebra and arithmetic. The main goal is to ensure you have enough scientific literacy to understand common physical theories, to be able to both appreciate and criticize them.

The design for this course is heavily inspired by the ``flipped-classroom" pioneered by Prof. Eric Mazur at Harvard University and currently implemented at many institutions, including the physics department at Cornell University. The idea is that you will be first introduced to concepts \textit{before} class, through preparatory readings and quizzes. Class time will be dedicated to a review of concepts, and discussion groups and activities to solidify the ideas. My role as instructor will be to provide a brief review, answer questions, and then facilitate discussions and engaging activities such as cooperative problem solving. In such a structure, participation is critical and will be a part of the final grade; you will not be penalized for asking questions or getting ``wrong" answers in discussions, you will be rewarded for being actively engaged.\newline

\noindent \textbf{Prerequisites}

This course will involve elementary algebra and arithmetic (specifically ratios, squared numbers, negative numbers, managing an equation which has 1 or 2 unknown variables, finding areas of simple shapes such as circles, triangles and rectangles). No prior knowledge of physics or advanced mathematics such as trigonometry or calculus is required. We will slowly introduce the required math through simple equations (such as $5x = 15$) for every concept, as well as work with invention tasks where you will discover the math yourself through intuitive examples.\newline


\noindent \textbf{Texts}

\textit{Light and Matter} by Benjamin Crowell will be our primary text, and \textit{College Physics} by Kim Dirks and Manjula Sharma will be use to supplement many of the chapters for those more comfortable with mathematical formalism.\newline

 \clearpage
 \noindent \textbf{Grade Distribution}\newline
A final grade out of 100 points will be given, broken down in the following way:
\begin{center}
\begin{tabularx}{0.8\textwidth}{l X}
15 Points & Class Participation \newline \textit{Participation in class discussion and activities such as  in-class cooperative problems, completing pre-class quizzes. A vital component to successfully completing this course is reading material before class, and so for every reading there will be a pre-class quiz (on what was in the reading, not as a test of the material itself).} \\
45 Points & Assignments \newline \textit{To be completed bi-weekly (for a total of 6 assignments). Discussion and cooperation on assignments is strongly encouraged, however each student is responsible for their own work; Plagiarism will result in 0 points. If you collaborate on an assignment, please indicate with whom by including their names in a note at the end of the assignment. The assignments for this semester are: \newline \hspace{0.25in}1) Philosophy of Science and Units \newline 2) Kinematics (equations and graphs) \newline 3) Forces, Energy and Momentum \newline 4) Atomic/Particle Theory, Light and Fields \newline 5) Special and General Relativity \newline 6) Probability and Quantum Mechanics} \\
40 Points & Partial exams \newline \textit{There will be 2 partial exams, one at the end of each section and worth 20 points each. Furthermore, the tests will be co-operative; You will take the test first individually (for 45 minutes) then pause for a 5 minute break, and then re-take the same test again in groups of 3 (for another 45 minutes). Your final grade for the test will either be solely the individual grade or a combination of the individual and group component, which ever is greater. All tests will be closed-book, however you will be allowed 1 double-sided cheat-sheet, on which you may write whatever you wish and use as a reference during the test.} \\
\end{tabularx}
\end{center}

\noindent \textbf{Course Outline}

The course is divided into 2 main sections, each ending in a partial exam. I will make every effort to adhere to this schedule, however some deviations may be necessary if certain topics are more challenging than expected. At the end of each section, we will have a review class to cover challenging topics before the partial test.

\begin{itemize}

\item In the first part of this course, we will describe motion and it's interplay with \textit{forces}. We will introduce concepts like \textit{velocity}, \textit{acceleration}, \textit{energy} and \textit{momentum} and how they relate. We will describe them numerically with equations, visually with graphs and conceptually through thought experiments and invention activities.
\item In the second part of this course, we will look at atomic ant particle theory, and explore the paradigm-shifting theories of relativity and quantum mechanics. We will look at weird and exotic things like black holes and cosmic evolution, as well as completely counter-intuitive notions such as wave-particle duality and fields of force.


\end{itemize}

\noindent \begin{tabularx}{\textwidth}[c]{| c X |}
	\hline
	\multicolumn{2}{|c|}{\textbf{\large Part 1}} \\ \hline
	\textit{Week 1} & \textbf{Philosophy of Science} \\ Feb. 2 & What does it mean to \textit{be scientific}? How do theory and ideology differ? What makes a theory scientific? At its core, science is a way of thinking, an attitude we adopt when we try to understand and/or predict the behavior of something. We will discuss the scientific method, the cycle between theory and experiment, and analyze its strengths and weaknesses. We will look at the heliocentric (Sun-centered) vs. geocentric (Earth-centered) view of the solar system, specifically discussing the kinds of experiments we can design to test these models. 
	\newline In class reading: M. Fuller, \textit{Is Science an Ideology?} Philosophy Now \textbf{15} (1996)\\ \hline
	\textit{Week 2} & \textbf{Units} \\ Feb. 9 & Reading: Chapter 0 (Sections 5, 8 and 9) of \textit{Light and Matter} \newline In physics, we want to quantitatively describe the world around us. In order to do this, how do we measure quantities? How can we compare measurements, if two people measure things differently? In this class we will discuss \textit{units}, what they are and how we can convert between them. We will introduce the concept of \textit{scientific notation}. From this, we will engage in an invention task that will introduce the notion of ratios, in order to characterize things like \textit{speed} and \textit{velocity}. \newline Math Concepts: ratios, powers of 10, positive and negative numbers \\ \hline
	\textit{Week 3} & \textbf{Kinematics} \\ Feb. 16 & Reading: Chapter 2 (Sections 1, 2 and 5) of \textit{Light and Matter}, Chapter 2 (Sections 1 to 5) of \textit{College Physics}, Chapter 3 (Sections 1 to 4, 6) of \textit{Light and Matter} \newline Everywhere we look things are in motion. We will look at ways of characterizing this motion: Using \textit{coordinate systems} to describe space, we will look at \textit{position}, \textit{velocity} and \textit{acceleration} to describe how objects move within it. Furthermore, while we live in a 3D world, we will confine ourselves to considering motion only in 1 dimension. We will work on an invention task centered on acceleration.\newline Math Concepts: ratios, powers of 10, variables ($x$, $y$, $t$, etc.), algebra ($5x=15$), squared numbers ($x^2$) \newline \textbf{DUE: Homework 1}\\  \hline
	\textit{Week 4} & \textbf{Graphs} \\ Feb. 23 & Reading: Chapter 2 (Section 3) of \textit{Light and Matter}, Chapter 2 (Section 8) of \textit{College Physics} \newline Graphs, diagrams, figures and pictures all all very useful tools to help us visualize and take in lots of information at once. We will use graphs to visually connect \textit{displacement}, \textit{velocity} and \textit{acceleration}, to describe in a still image an object's motion in time and how all these concepts relate to one another. This will be the most math-heavy class all semester, where we will find \textit{areas} under graphs, looks at \textit{slopes}, and connect these things to physical quantities. \newline Math Concepts: ratios, variables, areas, \textit{slopes} (no trigonometry), algebra ($5x=15$), squared numbers~($x^2$)\\ \hline
	\textit{Week 5} & \textbf{Forces} \\ March 2 & Reading: Handout on \textit{4 Fundamental Forces of Nature}, Chapter 4 of \textit{Light and Matter} \newline So far, we have been talking about motion without considering its cause. \textit{Forces} cause changes in motion (making things slow down, speed up, stop, push and pull on one another). We will look at the four fundamental forces of nature, build up to \textit{contact forces} (like the ones we experience everyday) and quantify forces with \textit{Newtons}. \newline Math Concepts: ratios, algebra ($5x=15$). \newline \textbf{DUE: Homework 2}\\ \hline
\end{tabularx}
\noindent \begin{tabularx}{\textwidth}[c]{| c X |}
\hline
	\textit{Week 6} & \textbf{Energy and Momentum} \\ March 16 & Reading: Handout (can skip section 3), Chapter 13 (Section 1) of \textit{Light and Matter} Chapter 7 (Sections 1 to 6) of \textit{College Physics} \newline We will discuss Noether's theorem, relating symmetries of nature to conserved quantities like \textit{energy} and \textit{momentum}. We will relate this quantities to forces, and introduce the concept of \textit{work} describing the transfer of energy. We will look at different forms of energy: kinetic (energy of motion), potential (``stored" energy and its relation to fundamental forces), and as mass itself (through the mass-energy equivalence, the famous $E=mc^2$). Energy and work will also be introduced in class through an invention activity.\newline Math Concepts: ratios, algebra ($5x = 15$), squared numbers~($x^2$), negative numbers\\ \hline
	\textit{Week 7} & \textbf{Review} \\ March 23 & This class will be dedicate to a review of the material from weeks 1-6. \newline \textbf{DUE: Homework 3}\\ \hline
	\textit{Week 8} & \textbf{First Partial Test} \\ March 30 & This test will cover material from the first part of the course.\\ \hline
	\end{tabularx}
\vspace{0.1in}

\noindent \begin{tabularx}{\textwidth}[c]{| c X |}
\hline
	\multicolumn{2}{|c|}{\textbf{\large Part 2}} \\ \hline
	\textit{Week 9} & \textbf{Atomic Theory and Light} \\ April 6 & Reading: Chapter 17 (Section 1) of \textit{Light and Matter}, Chapter 19 (Section 3 and 4) of \textit{Light and Matter}, Chapter 26 (Section 1, 4) of \textit{Light and Matter} \newline We will discuss atomic and particle theory, looking at particles like \textit{electrons}, \textit{protons}, \textit{neutrons}, \textit{quarks} and how they come together to form the atom. We will look at the periodic table, and use it as a guide to understanding the differences between atoms (for things like \textit{elements} and \textit{isotopes}). We will then turn our attention to a different kind of particle: \textit{photons} (particles of light). We will discuss light; As a particle (\textit{photon}) and as a wave (as well as the \textit{electro-magnetic spectrum} of all light-waves).\newline Math Concepts: ratios\\ \hline
	\textit{Week 10} & \textbf{Fields of Force} \\ April 13 & Reading: Chapter 22 (Sections 1 to 3) of \textit{Light and Matter} \newline \textit{How} do forces act on objects? The Earth and the Sun are separated by vast amounts of space - exactly \textit{how} to they ``communicate" in order to be attracted to one another? Here we will discuss \textit{fields} of force, specifically for gravity and the electric force. We will introduce the concept of a \textit{vector}, to describe the field at every point in space, look at superposition of fields (when there are lots of charges, or lots of masses). We will work on an invention activity that focuses on developing fields.\newline Math concepts: ratios, vectors\\ \hline
	\end{tabularx}

\noindent \begin{tabularx}{\textwidth}[c]{| c X |}
\hline
	\textit{Week 11} & \textbf{Special Relativity} \\ April 20 & Reading: Chapter 23 (Section 1) of \textit{Light and Matter}, Chapter 28 (Sections 1, 2, 3) of \textit{College Physics} \newline We live our lives on a very unique length scale; We are minuscule when compared to the Earth, moving at speeds almost non-existent when compared to light. What happens when we consider physics outside of our realm - massive objects moving very, very fast? Intuition is no longer our guide here, as time \textit{dilates} and lengths \textit{contract}, and events that are instantaneous to one observer occur one after another to some other observer. We will look at Einstein's postulates and reference frames.\newline Math concepts: Little to no math will be used.\newline \textbf{DUE: Homework 4}\\ \hline
	\textit{Week 12} & \textbf{General Relativity and \textit{Cosmology}} \\ April 25 & Reading: Chapter 27 of \textit{Light and Matter} \newline Space is not a square, rigid thing. \textit{Spacetime} bends and distorts, influencing how objects travel through space: Objects bend and warp \textit{spacetime}. We will look at ``gravitational wells" and how they influence light and matter (the \textit{Doppler effect}), and use this framework to discuss black holes and the \textit{cosmological model} of the big bang.\\ \hline
	\textit{Week 13} & \textbf{Probability and Quantum Mechanics} \\ May 4 & Reading: Chapter 33 (Sections 1, 2, 4) of \textit{Light and Matter}, Chapter 34 (Section 1 to 3) of \textit{Light and Matter}, Chapter 35 (Section 1) of \textit{Light and Matter} \newline If we knew the exact location, velocity and all the forces acting on everything in the universe, then could we predict the future (is the universe \textit{deterministic})? At the atomic scale, all evidence points to \textit{no}. A lot of behavior is described in terms of probabilities. Even stranger, according to the \textit{uncertainty principle} we \textit{cannot} know the exact location and velocity of an object! In this class we will discuss probabilities, and how it relates to quantum mechanics, specifically through radioactive decay. We will talk about the \textit{wave-particle duality}, specifically how light and electrons are both waves and particles. \newline Math Concepts: ratios, reading graphs \newline \textbf{DUE: Homework 5}\\ \hline
	\textit{Week 14} & \textbf{Review} \\ May 11 & This class will be dedicated to a review of material from weeks 9-13.\newline \textbf{DUE: Homework 6}\\ \hline
	\textit{Week 15} & \textbf{Second Partial Test} \\ May 18 & The second partial test will cover material from weeks 9-13 (and NOT include material from the first part of the course).\\ \hline
\end{tabularx}
	

\clearpage

A portion of class grades are dedicated to class participation, and are divided into 3 main components with 5 points each, for a total of 15 points (out of 100 for the total grade). \newline

\noindent\begin{tabularx}{\textwidth}[c]{|X|X|X|X|}
\hline
 \multicolumn{1}{|c|}{\textbf{Component}} & \multicolumn{1}{c|}{\textbf{4-5 Points}} &  \multicolumn{1}{c|}{\textbf{2-3 Points}} &  \multicolumn{1}{c|}{\textbf{0-1 Points}}\\
\hline
\textbf{Listening} \newline Ability to be focused and respectful during class time. & Paying respectful attention to the instructor, teaching assistant and fellow peers. & Paying attention to the instructor, teaching assistant and fellow peers most of the time, with some periods of visible distraction or lack of interest. & Being mostly inattentive or disconnected from class discussion, or disrespectful and disruptive of other people's contributions.\\
\hline
\textbf{Preparation} \newline Ability to adequately prepare for class (e.g. doing assigned homework, taking steps to address challenging sections, etc.) & Coming to class having read and completed the assigned material and well prepared to work on the class topic. Identifying and taking immediate action to address particularly challenging sections. & Coming to class somewhat familiar with the assigned material but without having done all the readings or assignments, making limited effort to address flaws in comprehension. & Coming to class with little or no knowledge of the assigned material, making almost no effort to address flaws in comprehension.\\
\hline
\textbf{Engagement} \newline Ability to participate in class in a way which is meaningful and indicates personal reflection (e.g. asking and answering questions, presenting and challenging arguments, etc.) & Taking active part in class activities, and contributing in a way that shows clear evidence of individual thought. & Occasionally taking part in class activities, contributing minimally to discussions. & Almost never taking part in class activities, with little to no contributions to class discussions.\\
\hline
\end{tabularx}

\end{document}