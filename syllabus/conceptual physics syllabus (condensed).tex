\documentclass[12pt]{article}
\usepackage[letterpaper, margin=0.75in]{geometry}
\usepackage{tabularx}
\usepackage{graphicx}
\usepackage{titling}


\pretitle{\begin{center}\LARGE\includegraphics[width=10cm]{../images/logo.png}\\[\bigskipamount]}
\posttitle{\end{center}}

\begin{document}

\title{Conceptual Physics Syllabus \\ Condensed Version}

\author{Instructor: Ms. Quinn} 
	
\maketitle


\noindent \textbf{Changes to Course Outline}

In order to accommodate ending on \textbf{April 30th}, three classes needed to be removed. Do do so, the following changes were made:
\begin{itemize}
	\item Remove the 2 review classes.
	\item Condense relativity section (special and general) from 2 classes to one. As a result, there will be 2 reading quizzes due that week instead of 1.
	\item Bring up the dates of the partial tests
	\item Questions on the first partial test will be slightly different, so that solutions can be handed out without impacting the other students (who will be taking it later).
	\item Have probability and quantum mechanics be the first thing covered in part 2, rather than the last. In doing so, 3 classes in part 2 will align (Atomic theory and Light, Fields of Force, and Special Relativity).
	\item Homework 3 and Homework 5 will now be due the day of the partial tests. As a result, no solutions for those homeworks will be provided before the partial test which covers that material. As a compromise, formal solutions to the in-class problems for the missed material will be provided.
	\item Homework 6 will now be moved to be due before homework 4 (since it covers probability and quantum mechanics). It will also contain slightly different problems, so that solutions can be handed out without impacting the other students (who will be working on Homework 6 later in the semester).
\end{itemize}

\clearpage
\noindent As a comparison, the following table illustrates the changes:
\vspace{0.1in}

\noindent \begin{tabularx}{\textwidth}[c]{c X X}
	\hline
	\textbf{Date} & \textbf{Original Timeline} & \textbf{Shifted Timeline} \\
	\hline
	\textit{Week 1} & Philosophy of Science & Philosophy of Science \\
	{\small Feb. 2} & & \\
	\hline
	\textit{Week 2} & Units & Units \\
	{\small Feb. 9} & RQ 1 & RQ 1 \\
	\hline 
	\textit{Week 3} & Kinematics & Kinematics \\
	{\small Feb. 16} & RQ 2, HW 1 & RQ 2, HW 1 \\
	\hline
	\textit{Week 4} & Graphs & Graphs \\
	{\small Feb. 23} & RQ 3 & RQ 3 \\
	\hline
	\textit{Week 5} & Forces & Forces \\
	{\small March 2} & RQ 4, HW 2 & RQ 4, HW 2 \\
	\hline
	\textit{Week 6} & Energy and Momentum & Energy and Momentum \\
	{\small March 16} & RQ 5 & RQ 5 \\
	\hline
	\textit{Week 7} & Review & \textbf{First Partial Test} \\
	{\small March 23} & HW 3 & HW 3 \\
	\hline
	\textit{Week 8} & \textbf{First Partial Test} & Probability and Quantum Mechanics \\
	{\small March 30} & & RQ 10 \\
	\hline
	\textit{Week 9} & Atomic Theory and Light & Atomic Theory and Light \\
	{\small April 6} & RQ 6 & RQ 6, HW 6 \\
	\hline
	\textit{Week 10} & Fields of Force & Fields of Force \\
	{\small April 13} & RQ 7 & RQ 7 \\
	\hline
	\textit{Week 11} & Special Relativity & Relativity (Special and General) \\
	{\small April 20} & RQ 8, HW 4 & RQ 8, RQ 9, HW 4 \\
	\hline
	\textit{Week 12} & General Relativity and \textit{Cosmology} & \textbf{Second Partial Test} \\
	{\small April 25} & RQ 9 & HW 5 \\
	\hline
	\textit{Week 13} & Probability and Quantum Mechanics & \\
	{\small May 4} & RQ 10, HW 5 & \\
	\hline
	\textit{Week 14} & Review & \\
	{\small May 11} & HW 6 & \\
	\hline
	\textit{Week 15} & \textbf{Second Partial Test} & \\
	{\small May 18} & & \\
	\hline
	
\end{tabularx}

\noindent \begin{tabularx}{\textwidth}[c]{| X |}
	\hline
	{\textbf{\large Part 1}} \\ \hline
	\textbf{Philosophy of Science} \newline What does it mean to \textit{be scientific}? How do theory and ideology differ? What makes a theory scientific? At its core, science is a way of thinking, an attitude we adopt when we try to understand and/or predict the behavior of something. We will discuss the scientific method, the cycle between theory and experiment, and analyze its strengths and weaknesses. We will look at the heliocentric (Sun-centered) vs. geocentric (Earth-centered) view of the solar system, specifically discussing the kinds of experiments we can design to test these models. \newline In class reading: M. Fuller, \textit{Is Science an Ideology?} Philosophy Now \textbf{15} (1996)\\ \hline
	\textbf{Units} \newline Reading: Chapter 0 (Sections 5, 8 and 9) of \textit{Light and Matter} \newline In physics, we want to quantitatively describe the world around us. In order to do this, how do we measure quantities? How can we compare measurements, if two people measure things differently? In this class we will discuss \textit{units}, what they are and how we can convert between them. We will introduce the concept of \textit{scientific notation}. From this, we will engage in an invention task that will introduce the notion of ratios, in order to characterize things like \textit{speed} and \textit{velocity}. \newline Math Concepts: ratios, powers of 10, positive and negative numbers \\ \hline
	\textbf{Kinematics} \newline Reading: Chapter 2 (Sections 1, 2 and 5) of \textit{Light and Matter}, Chapter 2 (Sections 1 to 5) of \textit{College Physics}, Chapter 3 (Sections 1 to 4, 6) of \textit{Light and Matter} \newline Everywhere we look things are in motion. We will look at ways of characterizing this motion: Using \textit{coordinate systems} to describe space, we will look at \textit{position}, \textit{velocity} and \textit{acceleration} to describe how objects move within it. Furthermore, while we live in a 3D world, we will confine ourselves to considering motion only in 1 dimension. We will work on an invention task centered on acceleration.\newline Math Concepts: ratios, powers of 10, variables ($x$, $y$, $t$, etc.), algebra ($5x=15$), squared numbers ($x^2$)\\  \hline
	\textbf{Graphs} \newline Reading: Chapter 2 (Section 3) of \textit{Light and Matter}, Chapter 2 (Section 8) of \textit{College Physics} \newline Graphs, diagrams, figures and pictures all all very useful tools to help us visualize and take in lots of information at once. We will use graphs to visually connect \textit{displacement}, \textit{velocity} and \textit{acceleration}, to describe in a still image an object's motion in time and how all these concepts relate to one another. This will be the most math-heavy class all semester, where we will find \textit{areas} under graphs, looks at \textit{slopes}, and connect these things to physical quantities. \newline Math Concepts: ratios, variables, areas, \textit{slopes} (no trigonometry), algebra ($5x=15$), squared numbers~($x^2$)\\ \hline
	\textbf{Forces} \newline Reading: Handout on \textit{4 Fundamental Forces of Nature}, Chapter 4 of \textit{Light and Matter} \newline So far, we have been talking about motion without considering its cause. \textit{Forces} cause changes in motion (making things slow down, speed up, stop, push and pull on one another). We will look at the four fundamental forces of nature, build up to \textit{contact forces} (like the ones we experience everyday) and quantify forces with \textit{Newtons}. \newline Math Concepts: ratios, algebra ($5x=15$).\\ \hline
\end{tabularx}

\noindent \begin{tabularx}{\textwidth}[c]{| X |}
\hline
	\textbf{Energy and Momentum} \newline Reading: Handout (can skip section 3), Chapter 13 (Section 1) of \textit{Light and Matter} Chapter 7 (Sections 1 to 6) of \textit{College Physics} \newline We will discuss Noether's theorem, relating symmetries of nature to conserved quantities like \textit{energy} and \textit{momentum}. We will relate this quantities to forces, and introduce the concept of \textit{work} describing the transfer of energy. We will look at different forms of energy: kinetic (energy of motion), potential (``stored" energy and its relation to fundamental forces), and as mass itself (through the mass-energy equivalence, the famous $E=mc^2$). Energy and work will also be introduced in class through an invention activity.\newline Math Concepts: ratios, algebra ($5x = 15$), squared numbers~($x^2$), negative numbers\\ \hline
	\end{tabularx}
\vspace{0.1in}

\noindent \begin{tabularx}{\textwidth}[c]{| X |}
\hline
	{\textbf{\large Part 2}} \\ \hline
	\textbf{Probability and Quantum Mechanics} \newline Reading: Chapter 33 (Sections 1, 2, 4) of \textit{Light and Matter}, Chapter 34 (Section 1 to 3) of \textit{Light and Matter}, Chapter 35 (Section 1) of \textit{Light and Matter} \newline If we knew the exact location, velocity and all the forces acting on everything in the universe, then could we predict the future (is the universe \textit{deterministic})? At the atomic scale, all evidence points to \textit{no}. A lot of behavior is described in terms of probabilities. Even stranger, according to the \textit{uncertainty principle} we \textit{cannot} know the exact location and velocity of an object! In this class we will discuss probabilities, and how it relates to quantum mechanics, specifically through radioactive decay. We will talk about the \textit{wave-particle duality}, specifically how light and electrons are both waves and particles. \newline Math Concepts: ratios, reading graphs\\ \hline
	\textbf{Atomic Theory and Light} \newline Reading: Chapter 17 (Section 1) of \textit{Light and Matter}, Chapter 19 (Section 3 and 4) of \textit{Light and Matter}, Chapter 26 (Section 1, 4) of \textit{Light and Matter} \newline We will discuss atomic and particle theory, looking at particles like \textit{electrons}, \textit{protons}, \textit{neutrons}, \textit{quarks} and how they come together to form the atom. We will look at the periodic table, and use it as a guide to understanding the differences between atoms (for things like \textit{elements} and \textit{isotopes}). We will then turn our attention to a different kind of particle: \textit{photons} (particles of light). We will discuss light; As a particle (\textit{photon}) and as a wave (as well as the \textit{electro-magnetic spectrum} of all light-waves).\newline Math Concepts: ratios\\ \hline
	\textbf{Fields of Force} \newline Reading: Chapter 22 (Sections 1 to 3) of \textit{Light and Matter} \newline \textit{How} do forces act on objects? The Earth and the Sun are separated by vast amounts of space - exactly \textit{how} to they ``communicate" in order to be attracted to one another? Here we will discuss \textit{fields} of force, specifically for gravity and the electric force. We will introduce the concept of a \textit{vector}, to describe the field at every point in space, look at superposition of fields (when there are lots of charges, or lots of masses). We will work on an invention activity that focuses on developing fields.\newline Math concepts: ratios, vectors\\ \hline
	\end{tabularx}

\noindent \begin{tabularx}{\textwidth}[c]{| X |}
\hline
	\textbf{Relativity} \newline Reading (Special Relativity): Chapter 23 (Section 1) of \textit{Light and Matter}, Chapter 28 (Sections 1, 2, 3) of \textit{College Physics} \newline Reading (General Relativity): Chapter 27 of \textit{Light and Matter} \newline We live our lives on a very unique length scale; We are minuscule when compared to the Earth, moving at speeds almost non-existent when compared to light. What happens when we consider physics outside of our realm - massive objects moving very, very fast? Intuition is no longer our guide here, as time \textit{dilates} and lengths \textit{contract}, and events that are instantaneous to one observer occur one after another to some other observer. We will look at Einstein's postulates and reference frames.\newline Math concepts: Little to no math will be used.\\ \hline
\end{tabularx}

\end{document}