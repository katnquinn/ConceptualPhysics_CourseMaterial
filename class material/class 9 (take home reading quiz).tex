\documentclass[addpoints,12pt]{exam}
%\documentclass[12pt]{article}
\usepackage[letterpaper, margin=0.75in]{geometry}
\usepackage{graphicx}
\usepackage{enumitem}
\usepackage{booktabs}

\begin{document}
\footer{}{Page \thepage\ of \numpages}{}

\begin{flushright}
\makebox[0.5\textwidth]{\large Name:\enspace\hrulefill}
\vspace{0.2in}

\makebox[0.5\textwidth]{\large Date:\enspace\hrulefill}
\end{flushright}

\begin{center}
\includegraphics[width=10cm]{../images/logo.png}
\end{center}

\begin{center}
\noindent{\LARGE Conceptual Physics \\ Reading Quiz 7}
\end{center}

\noindent\begin{large}\textbf{Due Date: April 13 (before class)}\end{large}
\vspace{0.2in}

This reading assignment covers the material that we will discuss and work on next week. Our class activities will \textbf{assume} that you have read the assigned material, therefore it is \textbf{very important} that you do so to get the most out of the class!

\begin{enumerate}


	\item \textit{Light and Matter} Chapter 22 (Section 1 to 3)
	
	Fields of force are a very strange, counter-intuitive way of modeling how particles interact with each other. We will spend a lot of time in class going over different cases of fields, drawing and interpreting them. When the text discusses the \textit{superposition principle} for fields, it uses the terminology of \textit{vector addition}. I in no way expect you to be familiar with this, and we will discuss it at length next class to add fields together when there are multiple objects.
	
	We will be qualitatively drawing field, not quantitatively calculating them - so don't worry about how to numerically add fields with different angles (and ignore all equations with $\sin\theta$ and $\cos\theta$). In particular, pay attention to figure \textit{m} which draws the field lines around \textit{Dipoles}.
	
	The section that discusses voltage is useful for the qualitative relationship between charges and potential energy, but we will not be using any of the equations.
	
\end{enumerate}

As part of the reading, please complete the pre-class quiz \textit{before} coming to class. They will be collected at the very beginning.
 
\clearpage

\begin{flushright}
Score: \hspace{0.2in} / \numpoints ~ points
\end{flushright}

\begin{questions}

\question[1]
Based on our current understanding of physics, is instantaneous action at a distance possible?
\fillwithlines{0.5in}

\question[1]
Define a gravitational field.
\fillwithlines{0.5in}

\question[2]
What is the difference between a field \textit{source} and a field \textit{sink}?
\fillwithlines{0.5in}

\question[1]
Define an electric field.
\fillwithlines{0.5in}

\question[2]
Is a positive charge a source or a sink? Is a negative charge a source or a sink?
\fillwithlines{0.5in}

\question[3]
Below is a positive and negative charge, forming a dipole. Draw the field lines around them.
\vspace{1.5in}
\begin{center}
\includegraphics[]{../images/dipole.png}
\end{center}



\end{questions}


\end{document}