\documentclass[addpoints,12pt]{exam}
%\documentclass[12pt]{article}
\usepackage[letterpaper, margin=0.75in]{geometry}
\usepackage{graphicx}
\usepackage{enumitem}
\usepackage{booktabs}

\begin{document}
\footer{}{Page \thepage\ of \numpages}{}

\begin{flushright}
\makebox[0.5\textwidth]{\large Name:\enspace\hrulefill}
\vspace{0.2in}

\makebox[0.5\textwidth]{\large Date:\enspace\hrulefill}
\end{flushright}

\begin{center}
\includegraphics[width=10cm]{../images/logo.png}
\end{center}

\begin{center}
\noindent{\LARGE Conceptual Physics \\ Reading Quiz 10}
\end{center}

\noindent\begin{large}\textbf{Due Date: May 4 (before class)} \end{large}
\vspace{0.2in}

This reading assignment covers the material that we will discuss and work on next week. Our class activities will \textbf{assume} that you have read the assigned material, therefore it is \textbf{very important} that you do so to get the most out of the class!

\begin{enumerate}


	\item \textit{Light and Matter} Chapter 33 (Sections 1, 2 and 4)
	
	Pay attention to when probabilities are added and multiplied. In section 4, during the discussion of radioactive decay, what's most important is that you understand the notion of a \textit{half-life}. Since we do not have calculators, we will not be using the mathematical operation of $\log$ that they do in example 3. You can skip the portion that derives this relation (rate of decay).
	
	\item \textit{Light and Matter} Chapter 34 (Sections 1 to 3)
	
	The key point in this chapter is the wave/particle duality as applied to light, and the main equation to focus on is the energy of a particle of light.
	
	\item \textit{Light and Matter} Chapter 35 (Section 1)
	
	

	
\end{enumerate}

As part of the reading, please complete the pre-class quiz \textit{before} coming to class. They will be collected at the very beginning.
 
\clearpage

\begin{flushright}
Score: \hspace{0.2in} / \numpoints ~ points
\end{flushright}


\noindent Questions for \textit{Light and Matter} Chapter 33

\begin{questions}

\question[1]
What is the source of most of Earth's heat?
\fillwithlines{0.5in}

\question[1]
If something has probability 0 of occurring, is it something impossible or something that will definitely happen?
\fillwithlines{0.5in}

\question[1]
What is the law of independent probabilities?
\fillwithlines{1in}

\question[1]
What is the rule for calculating averages?
\fillwithlines{1in}

\end{questions}


\noindent Questions for \textit{Light and Matter} Chapter 34

\begin{questions}

\question[1]
What is a photon?
\fillwithlines{0.5in}

\question[1]
How is the energy of a photon related to its frequency ($f$) and Planck's constant ($H$)?
\fillwithlines{0.5in}

\end{questions}

\noindent Questions for \textit{Light and Matter} Chapter 35

\begin{questions}

\question[4]
While both light and electrons (and protons, and neutrons) exhibit wave-particle duality, there are some important ways in which light and electrons differ. List 4.
\fillwithlines{1in}


\end{questions}


\end{document}