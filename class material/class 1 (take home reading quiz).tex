\documentclass[addpoints,12pt]{exam}
%\documentclass[12pt]{article}
\usepackage[letterpaper, margin=0.75in]{geometry}
\usepackage{graphicx}
\usepackage{enumitem}
\usepackage{booktabs}

\begin{document}
\footer{}{Page \thepage\ of \numpages}{}

\begin{flushright}
\makebox[0.5\textwidth]{\large Name:\enspace\hrulefill}
\vspace{0.2in}

\makebox[0.5\textwidth]{\large Date:\enspace\hrulefill}
\end{flushright}

\begin{center}
\includegraphics[width=10cm]{../images/logo.png}
\end{center}

\begin{center}
\noindent{\LARGE Conceptual Physics \\ Reading Quiz 1}
\end{center}

\noindent\begin{large}\textbf{Due Date: Feb. 9 (before class)}\end{large}
\vspace{0.2in}

This reading assignment covers the material that we will discuss and work on next week. Our class activities will \textbf{assume} that you have read the assigned material, therefore it is \textbf{very important} that you do so to get the most out of the class!
\begin{enumerate}
	\item \textit{Light and Matter}, Chapter 0 (Sections 5, 8 and 9)
	
	In this class, we will be using the metric system for the majority of our calculations (kilometers instead of miles, kilograms instead of pounds) and so this reading will be very useful if you are not familiar with the metric system.
	
	In section 8, we discuss the \textit{scientific notation}, where numbers are represented using powers of 10. Please read through this section carefully, and try to figure out how to go between numbers like $314$, $31.4\times 10$, $3.14\times 10^2$, $3140\times 10^{-1}$, $\frac{3140}{10}$, etc. This kind of notation is really weird, and so we will spend  a lot of time in class discussing how to use powers of 10, so it's OK if you don't have it mastered by next week.
	
	In section 9, we discuss conversion factors. You do not need to memorize any conversion factors - they will always be provided. Since we cannot use calculators in this course, I will provide rounded conversion factors to make calculations easier (1~inch = 2.5~cm instead of 2.54~cm, for example).
	
	If you have time, please feel free to read the rest of the sections in chapter 0, particularly the introduction where the author discusses the cycle of model and experimentation. If you are interested, we will be discussing forces week 5, so section 6 will give you a preview of things to come. Section 10 is not necessary at all for this course. Significant digits are essential in reporting measured data from experiments, however we will not be performing any experiments (and discussing error/uncertainty) and so this section is entirely unnecessary.
	
	
\end{enumerate}

As part of the reading, please complete the pre-class quiz \textit{before} coming to class. They will be collected at the very beginning.
 
\clearpage

\begin{flushright}
Score: \hspace{0.2in} / \numpoints ~ points
\end{flushright}

\begin{questions}

\question[1]
The metric system works with a single, consistent set of Greek and Latin prefixes that modify basic units. Each prefix stands for a power of what number?
\fillwithlines{0.5in}

\question[3]
What are the three most common metric prefixes and their meaning?
\fillwithlines{0.75in}

\question[1]
We no longer use the Earth's rotation to define the hour, minute and second. Instead, we rely on the number of vibrations of light waves emitted from what kind of atom?
\fillwithlines{0.5in}

\question[1]
The modern definition of a meter is the distance light travels in vacuum in how many seconds?
\fillwithlines{0.5in}

\question[1]
In what city is the ``mother of all kilograms" stored?
\fillwithlines{0.5in}

\question[1]
Scientific notation means writing a number in terms of what two things?
\fillwithlines{0.5in}

\question[1]
One year is equal to how many seconds?
\fillwithlines{0.5in}

\question[1]
Which of the following fraction is ``correct", in the sense that it is equal to 1? (circle the correct one)
\begin{itemize}
\item $\frac{10^3 kg}{1 g}$
\item $\frac{10^{-3} kg}{1 g}$
\end{itemize}

\end{questions}



\end{document}