\documentclass[addpoints,12pt]{exam}
%\documentclass[12pt]{article}
\usepackage[letterpaper, margin=0.75in]{geometry}
\usepackage{graphicx}
\usepackage{enumitem}
\usepackage{booktabs}

\begin{document}
\footer{}{Page \thepage\ of \numpages}{}

\begin{flushright}
\makebox[0.5\textwidth]{\large Name:\enspace\hrulefill}
\vspace{0.2in}

\makebox[0.5\textwidth]{\large Date:\enspace\hrulefill}
\end{flushright}

\begin{center}
\includegraphics[width=10cm]{../images/logo.png}
\end{center}

\begin{center}
\noindent{\LARGE Conceptual Physics \\ Reading Quiz 5}
\end{center}

\noindent\begin{large}\textbf{Due Date: March 16 (before class)}\end{large}
\vspace{0.2in}

This reading assignment covers the material that we will discuss and work on next week. Our class activities will \textbf{assume} that you have read the assigned material, therefore it is \textbf{very important} that you do so to get the most out of the class!
\begin{enumerate}

	\item Handout on \textit{Conservation of Mass and Energy} in the course reader. (All sections except 3)
	
	This reading introduces the fundamental concept of energy, and the different forms it can take. Pay attention to both the \textit{operational} definition as well as the \textit{mathematical} equations for kinetic and gravitational potential energy (for object on the surface of the Earth), since we will be using them next class. Section 7, on the equivalence of Mass and Energy, will be useful in Part 2 of this course when we talk about Einstein's theory of relativity, and quantum mechanics, but if you're short on reading time you can skip it.
	
	\item \textit{Light and Matter}, Chapter 13 (Section 1)
	
	After introducing energy in the previous reading, this section discusses how energy is transferred and turned from one form into another. Pay attention to the difference between positive and negative work, and how this relates to gaining and losing energy.
	
	\item OPTIONAL: \textit{College Physics}, Chapter 7 (Sections 1 to 6)
	
	Please make sure to read this last, since it introduces a lot of mathematical formalism for energy, momentum and work. Please ignore the trigonometry used in the equations (they talk about $\cos\theta$) for different angles, since we will not be considering work in multiple dimensions - and see if you can relate section 1 to section 1 from chapter 13 in \textit{Light and Matter} (ignore Section 1 if not). 
	
	In section 2, they talk about work as it relates to areas under graphs - again this is above the level we will talk about in class, but feel free to read if you want. The most important part of section 2 is equation 7.1, which talks about the work-energy theorem (and how the quantity $\frac{1}{2}mv^2$ relates to the kinetic energy of an object). 
	
	Pay close attention to section 3, which discusses gravitational potential energy - it gives a very useful example with roller coasters.
	
	In sections 4 and 5, the key point is the relationship between different kinds of forces (conservative and non-conservative) as well as different kinds of energy (kinetic, potential), which we will discuss more thoroughly in class. Ignore the calculations for energy of a spring.
	
	Section 6 introduces mathematical formalism to the topics in the previous readings (which is all more reason to read this section last).
	
	
	
\end{enumerate}

As part of the reading, please complete the pre-class quiz \textit{before} coming to class. They will be collected at the very beginning.
 
\clearpage

\begin{flushright}
Score: \hspace{0.2in} / \numpoints ~ points
\end{flushright}

\noindent Questions from handout in course reader on Conservation of Mass and Energy

\begin{questions}

\question[1]
Physicist Emmy Noether showed that every symmetry of nature leads to what?
\fillwithlines{0.5in}

\question[1]
What kind of energy (kinetic or gravitational potential) is associated with motion? What kind of energy (kinetic or gravitational potential) is associated with objects being separated from each other?
\fillwithlines{0.5in}

\question[1]
Does doubling an object's velocity double it's kinetic energy?
\fillwithlines{0.5in}

\question[1]
If you were to travel from sea level to the top of a mountain, you would experience a slight reduction in gravity. Why?
\fillwithlines{0.5in}
\end{questions}

\noindent Questions from \textit{Light and Matter} Chapter 13

\begin{questions}

\question[1]
Define \textit{work} in the physics context of energy.
\fillwithlines{0.5in}

\question[1]
How is work related to force and distance?
\fillwithlines{0.5in}

\end{questions}

\noindent Questions from \textit{College Physics} Chapter 7

\begin{questions}

\bonusquestion[1]
What is a conservative force?
\fillwithlines{0.5in}

\bonusquestion[1]
What kind of energy is associated with conservative forces?
\fillwithlines{0.5in}

\bonusquestion[1]
What is a non-conservative force?
\fillwithlines{0.5in}

\bonusquestion[1]
What kind of work is done by a non-conservative forces?
\fillwithlines{0.5in}


\end{questions}

\end{document}