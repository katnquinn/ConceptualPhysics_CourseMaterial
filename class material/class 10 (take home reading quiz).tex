\documentclass[addpoints,12pt]{exam}
%\documentclass[12pt]{article}
\usepackage[letterpaper, margin=0.75in]{geometry}
\usepackage{graphicx}
\usepackage{enumitem}
\usepackage{booktabs}

\begin{document}
\footer{}{Page \thepage\ of \numpages}{}

\begin{flushright}
\makebox[0.5\textwidth]{\large Name:\enspace\hrulefill}
\vspace{0.2in}

\makebox[0.5\textwidth]{\large Date:\enspace\hrulefill}
\end{flushright}

\begin{center}
\includegraphics[width=10cm]{../images/logo.png}
\end{center}

\begin{center}
\noindent{\LARGE Conceptual Physics \\ Reading Quiz 8}
\end{center}

\noindent\begin{large}\textbf{Due Date: April 20 (before class)}\end{large}
\vspace{0.2in}

This reading assignment covers the material that we will discuss and work on next week. Our class activities will \textbf{assume} that you have read the assigned material, therefore it is \textbf{very important} that you do so to get the most out of the class!

The authors for the two different books take 2 different approaches to explaining time dilation. The \textit{Light and Matter} people talk about Lorentz contractions and how they effect position versus time graph, skewing them to show how things change. The \textit{College Physics} people use intricate geometry in a specific example. I encourage you to think about how the two are similar/different.

\begin{enumerate}


	\item \textit{Light and Matter} Chapter 23 (Section 1)
	
	This chapter covers relativity and magnetism, however we will only discuss relativity next class. The author spends a lot of time talking about distorted graphs. If you are comfortable with graphs, I encourage you to pay attention to the explanations since they are quite insightful. If you are uncomfortable with graphs, it's OK - we will spend most of our time in class talking about time dilation and length contraction without them.
	
	The $\gamma$ factor provides a quantitative way of describing how time/space are dilated/contracted for observers moving with respect to each other. Since we do not a have access to calculators, we will not be computing this factor when solving problems.
	
	\item \textit{College Physics} Chapter 28 (Sections 1 to 3)
	
	Pay attention to the postulates and definitions in this section. In particular, the author's discuss a confusing notion: It is possible for an observer to say 2 events occur simultaneously, while another observer says one occurs before/after the other. You may want to read the section on simultaneity a couple times, and parse out step by step what's happening with the train example. We will go over it again in class.
	
\end{enumerate}

As part of the reading, please complete the pre-class quiz \textit{before} coming to class. They will be collected at the very beginning.
 
\clearpage

\begin{flushright}
Score: \hspace{0.2in} / \numpoints ~ points
\end{flushright}


\noindent Question for \textit{Light and Matter} Chapter 23
\begin{questions}

\question[1]
When two observers in different frames of reference observe one another, do they perceive the other one's perception of space and time to be normal or distorted?
\fillwithlines{0.5in}

\question[1]
What is time dilation?
\fillwithlines{0.5in}

\question[1]
What is length contraction?
\fillwithlines{0.5in}

\end{questions}

\noindent Question for \textit{College Physics} Chapter 28
\begin{questions}

\question[1]
What is an inertial frame of reference?
\fillwithlines{0.5in}

\question[1]
What is Einstein's first postulate of special relativity?
\fillwithlines{0.5in}

\question[1]
What is Einstein's second postulate?
\fillwithlines{0.5in}

\question[2]
What is proper time?
\fillwithlines{0.5in}

\question[2]
What is proper length?
\fillwithlines{0.5in}

\end{questions}


\end{document}