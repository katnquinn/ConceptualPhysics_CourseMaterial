\documentclass[addpoints,12pt]{exam}
%\documentclass[12pt]{article}
\usepackage[letterpaper, margin=0.75in]{geometry}
\usepackage{graphicx}
\usepackage{enumitem}
\usepackage{booktabs}

\begin{document}
\footer{}{Page \thepage\ of \numpages}{}

\begin{flushright}
\makebox[0.5\textwidth]{\large Name:\enspace\hrulefill}
\vspace{0.2in}

\makebox[0.5\textwidth]{\large Date:\enspace\hrulefill}
\end{flushright}

\begin{center}
\includegraphics[width=10cm]{../images/logo.png}
\end{center}

\begin{center}
\noindent{\LARGE Conceptual Physics \\ Reading Quiz 9}
\end{center}

\noindent\begin{large}\textbf{Due Date: April 25 (before class)} Note that this is a Wednesday.\end{large}
\vspace{0.2in}

This reading assignment covers the material that we will discuss and work on next week. Our class activities will \textbf{assume} that you have read the assigned material, therefore it is \textbf{very important} that you do so to get the most out of the class!

\begin{enumerate}


	\item \textit{Light and Matter} Chapter 27 (all sections)
	
	Pay special attention to how spacetime is effected by matter, and how light is effected by curved spacetime (and how this is different from Newtonian mechanics). We will spent a lot of time in class discussing this difference, since it is often used as a way of disproving classical Newtonian mechanics and is used as evidence of some very strange and counter-intuitive models.
	
	You don't need to know the equations for Doppler shifts, but do pay attention to how light is red and blue shifted due to gravitational fields.	

	
\end{enumerate}

As part of the reading, please complete the pre-class quiz \textit{before} coming to class. They will be collected at the very beginning.
 
\clearpage

\begin{flushright}
Score: \hspace{0.2in} / \numpoints ~ points
\end{flushright}


\begin{questions}

\question[1]
A triangle normally has angles that add up to $180^0$. In space with negative curvature, do the angles still add up to $180^0$, more than $180^0$ or less than $180^0$?
\fillwithlines{0.5in}

\question[1]
A triangle normally has angles that add up to $180^0$. In space with positive curvature, do the angles still add up to $180^0$, more than $180^0$ or less than $180^0$?
\fillwithlines{0.5in}

\question[1]
What is the equivalence principle?
\fillwithlines{0.5in}

\question[1]
When one is lower in a gravitational field, does time run faster or slower?
\fillwithlines{0.5in}

\question[1]
When light waves rise through a gravitational field, to they gain or lose energy?
\fillwithlines{0.5in}

\question[1]
What is the event horizon of a black hole?
\fillwithlines{0.5in}
\question[1]
Bases on data from the Hubble telescope, is our universe expanding, contracting, or staying constant?
\fillwithlines{0.5in}

\question[1]
Does general relativity describe space as a rigid grid, or as something that can curve and bend?
\fillwithlines{0.5in}

\question[1]
When did Penrose propose the ``cosmic censorship hypothesis"
\fillwithlines{0.5in}

\question[1]
What is the cosmic microwave background?
\fillwithlines{0.5in}

\end{questions}



\end{document}