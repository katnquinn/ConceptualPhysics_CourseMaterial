\documentclass[12pt]{article}
\usepackage[letterpaper, margin=0.75in]{geometry}
\usepackage{tabularx}
\usepackage{graphicx}
\usepackage{titling}


\begin{document}

\begin{center}
\includegraphics[width=10cm]{../images/logo.png}
\end{center}

\begin{center}
\noindent{\LARGE Conceptual Physics \\ Reading Questions for \textit{Is Science an Ideology?}}
\end{center}

The following are suggested questions, that you are free to answer or leave black, to help guide our discussion after the reading. Feel free to write down any additional notes, comments, thoughts, suggestions, criticisms, etc. that come to mind as you read through.

\begin{enumerate}
	\item In the face of contradicting evidence which disproves a scientific model or scientific theories, the models and theories must be changed. What should we do if we have nothing to replace a false model?
	\vspace{2in}
	\item At what point does should entire scientific field be discredited? For instance, in physics Newtonian mechanics was disproved and replaced with the more modern theory of Relativity. If this pattern persists, at what point is our modern field of physics disproved and in need of replacement (just as chemistry replaced the outdated, yet arguably scientific, practice of alchemy)?
	\vspace{2in}
	\clearpage
	
	\item We create scientific models to explain the world around us, and make accurate predictions. We are often faced with complete paradigm shifts, such as General Relativity (which has warping space, dilating time) and quantum mechanics (which describes the subatomic world not as an orderly, deterministic machine but as a set of random events and fluctuations). Yet there is still a sense in which science ``works", since we can make airplanes and cure diseases. Should we describe scientific progress as a march towards Truth (an understand of the ``reality", whatever that may be) or rather as a march towards better puzzle solving, or something else entirely?
	\vspace{2in}
	
	\item Research is performed by people. Should we view it in the same light as other human endeavors (such as, say, writing poetry)?
	\vspace{2in}
	
	\item We are in a room with desks and chairs. How do we know that? Most people, without argument, would agree that this is true. How do we ``know" that matter is made up of atoms, or that the Earth is orbiting the sun?
	\vspace{2in}
\end{enumerate}


\end{document}