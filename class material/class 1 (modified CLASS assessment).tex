\documentclass[addpoints,12pt]{exam}
%\documentclass[12pt]{article}
\usepackage[letterpaper, margin=0.75in]{geometry}
\usepackage{graphicx}
\usepackage{enumitem}
\usepackage{booktabs}
\usepackage{color}

\graphicspath{{../images/}}


\begin{document}

\footer{}{Page \thepage\ of \numpages}{}

\begin{flushright}
\makebox[0.5\textwidth]{\large Name:\enspace\hrulefill}
\vspace{0.2in}

\makebox[0.5\textwidth]{\large Date:\enspace\hrulefill}
\end{flushright}

\begin{center}
\includegraphics[width=10cm]{../images/logo.png}
\end{center}

\begin{center}
\noindent{\LARGE Conceptual Physics \\ Pre-Class Survey}
\end{center}

Welcome to conceptual physics! In order for me (the instructor) to better know your physics backgrounds, get a feel for your prior knowledge and more generally understand the class's attitude towards physics, I would like to spend 20-30 minutes on a survey called CLASS (the Colorado Learning Attitudes about Science Survey). Your answers to this will in no way impact your grade, they are purely to help me teach better.

Here are a number of statements that may or may not describe your beliefs about learning physics. You are asked to rate each statement by selecting a number between 1 and 5
\begin{center}\input{../images/scale.pdf_tex}\end{center}

The numbers mean the following:
\begin{enumerate}
\item Strongly Disagree
\item Disagree
\item Neutral
\item Agree
\item Strongly Agree
\end{enumerate}

Choose one of the above 5 choices to express your feeling about the statement. If you don't understand a statement, leave it black. If you have no strong opinion, or feel  a particular statement does not apply to you, choose 3.

%\noindent\begin{center}\includegraphics{../images/confidence.eps}\end{center}

\clearpage

\begin{enumerate}
\item A significant problem in learning physics is being able to memorize all the information I need to know.
\begin{center}\input{../images/scale.pdf_tex}\end{center}

\item When I am solving a physics problem, I try to decide what would be a reasonable value for the answer.
\begin{center}\input{../images/scale.pdf_tex}\end{center}

\item I think about the physics I experience in everyday life.
\begin{center}\input{../images/scale.pdf_tex}\end{center}

\item It is useful for me to do lots and lots of problems when learning physics.
\begin{center}\input{../images/scale.pdf_tex}\end{center}

\item After I study a topic in physics and feel that I understand it, I have difficulty solving problems on the same topic.
\begin{center}\input{../images/scale.pdf_tex}\end{center}

\item Knowledge in physics consists of many disconnected topics.
\begin{center}\input{../images/scale.pdf_tex}\end{center}

\item As physicists learn more, most physics ideas we use today are likely to be proven wrong.
\begin{center}\input{../images/scale.pdf_tex}\end{center}

\item When I solve a physics problem, I locate an equation that uses the variables  in the problem and plug in the values.
\begin{center}\input{../images/scale.pdf_tex}\end{center}

\item I find that reading the text in detail is a good way for me to learn physics.
\begin{center}\input{../images/scale.pdf_tex}\end{center}

\item There is usually only one correct approach to solving a physics problem.
\begin{center}\input{../images/scale.pdf_tex}\end{center}

\item I am not satisfied until I understand why something works the way it does.
\begin{center}\input{../images/scale.pdf_tex}\end{center}

\item I cannot learn physics if the teacher does not explain things well in class.
\begin{center}\input{../images/scale.pdf_tex}\end{center}

\item I do not expect physics equations to help my understanding of ideas; they are just for doing calculations.
\begin{center}\input{../images/scale.pdf_tex}\end{center}

\item I study physics to learn knowledge that will be useful in my life outside of school.
\begin{center}\input{../images/scale.pdf_tex}\end{center}

\item If I get stuck on my first try, I usually try to figure out a different way that works.
\begin{center}\input{../images/scale.pdf_tex}\end{center}

\item Nearly everyone is capable of understanding physics if they work at it.
\begin{center}\input{../images/scale.pdf_tex}\end{center}

\item Understanding physics basically means being able to recall something you've read or been shown.
\begin{center}\input{../images/scale.pdf_tex}\end{center}

\item There could be two different correct values for the answer to a physics problem if I use two different approaches.
\begin{center}\input{../images/scale.pdf_tex}\end{center}

\item To understand physics I discuss it with friends and other students.
\begin{center}\input{../images/scale.pdf_tex}\end{center}

\item I do not spend more than five minutes stuck on a physics problem before giving up or seeking help from someone else.
\begin{center}\input{../images/scale.pdf_tex}\end{center}

\item If I don't remember a particular equation needed to solve a problem on an exam, there's nothing much I can do to come up with it.
\begin{center}\input{../images/scale.pdf_tex}\end{center}

\item If I want to apply a method used for solving one physics problem to another problem, the problems must involve very similar situations.
\begin{center}\input{../images/scale.pdf_tex}\end{center}

\item In doing a physics problem, if my calculation gives a result very different from what I'd expect, I'd trust the calculation rather than going back through the problem.
\begin{center}\input{../images/scale.pdf_tex}\end{center}

\item In physics, it is important for me to make sense out of formulas before I can use them correctly.
\begin{center}\input{../images/scale.pdf_tex}\end{center}

\item I enjoy solving physics problems.
\begin{center}\input{../images/scale.pdf_tex}\end{center}

\item In physics, mathematical formulas express meaningful relationships among measurable quantities.
\begin{center}\input{../images/scale.pdf_tex}\end{center}

\clearpage

\item It is important for the government to approve new scientific ideas before they can be widely accepted.
\begin{center}\input{../images/scale.pdf_tex}\end{center}

\item Learning physics changes my ideas about how the world works.
\begin{center}\input{../images/scale.pdf_tex}\end{center}

\item To learn physics, I only need to memorize solutions to sample problems.
\begin{center}\input{../images/scale.pdf_tex}\end{center}

\item Reasoning skills used to understand physics can be helpful to me in everyday life.
\begin{center}\input{../images/scale.pdf_tex}\end{center}

\item Spending lots of time understanding where formulas come from is a waste of time.
\begin{center}\input{../images/scale.pdf_tex}\end{center}

\item I find carefully analyzing only a few problems in detail is a good way for me to learn physics.
\begin{center}\input{../images/scale.pdf_tex}\end{center}

\item I can usually figure out a way to solve physics problems.
\begin{center}\input{../images/scale.pdf_tex}\end{center}

\item The subject of physics has little relation to what I experience in the real world.
\begin{center}\input{../images/scale.pdf_tex}\end{center}

\item There are times I solve a physics problem more than one way to help my understanding.
\begin{center}\input{../images/scale.pdf_tex}\end{center}

\item To understand physics, I sometimes think about my personal experiences and relate them to the topic being analyzed.
\begin{center}\input{../images/scale.pdf_tex}\end{center}

\item It is possible to explain formula without mathematical formula.
\begin{center}\input{../images/scale.pdf_tex}\end{center}

\item When I solve a physics problem, I explicitly think about which physics ideas apply to the problem.
\begin{center}\input{../images/scale.pdf_tex}\end{center}

\item If I get stuck on a physics problem, there is not chance I'll figure it out on my own.
\begin{center}\input{../images/scale.pdf_tex}\end{center}

\item It is possible for physicists to carefully perform the same experiment and get two very different answers that are both correct.
\begin{center}\input{../images/scale.pdf_tex}\end{center}

\item When studying physics, I relate the important information to what I already know rather than just memorizing the way it is presented.
\begin{center}\input{../images/scale.pdf_tex}\end{center}

\end{enumerate}

	

\end{document}