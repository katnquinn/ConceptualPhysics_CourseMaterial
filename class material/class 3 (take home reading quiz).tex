\documentclass[addpoints,12pt]{exam}
%\documentclass[12pt]{article}
\usepackage[letterpaper, margin=0.75in]{geometry}
\usepackage{graphicx}
\usepackage{enumitem}
\usepackage{booktabs}

\begin{document}
\footer{}{Page \thepage\ of \numpages}{}

\begin{flushright}
\makebox[0.5\textwidth]{\large Name:\enspace\hrulefill}
\vspace{0.2in}

\makebox[0.5\textwidth]{\large Date:\enspace\hrulefill}
\end{flushright}

\begin{center}
\includegraphics[width=10cm]{../images/logo.png}
\end{center}

\begin{center}
\noindent{\LARGE Conceptual Physics \\ Reading Quiz 3}
\end{center}

\noindent\begin{large}\textbf{Due Date: Feb. 23 (before class)}\end{large}
\vspace{0.2in}

This reading assignment covers the material that we will discuss and work on next week. Our class activities will \textbf{assume} that you have read the assigned material, therefore it is \textbf{very important} that you do so to get the most out of the class!
\begin{enumerate}
	\item \textit{Light and Matter}, Chapter 2 (Section 3)
	
	In this section, pay special attention to how velocity relates to position and time. In particular, it is important to be able to read a graph: What does the line represent? If I have a position versus time graph, how to I find an object position at 1 second? At 2 seconds? How do I find the time at which an object has moved by 1 m?
	Graphs can take on different shapes. They can be straight or curvy lines, and they can be pointed up or pointed down. What do the different shapes mean?
	
	
	\item \textit{College Physics}, Chapter 2 (Section 8)
	
	This section is particularly mathy - it's OK if you don't know what all the equations mean.
	
	This section is particularly interesting because it relates how \textit{acceleration} connects to velocity versus time graphs, and how it effects position versus time graphs.
	
	
	
\end{enumerate}

As part of the reading, please complete the pre-class quiz \textit{before} coming to class. They will be collected at the very beginning.
 
\clearpage

\begin{flushright}
Score: \hspace{0.2in} / \numpoints ~ points
\end{flushright}

\noindent Questions from \textit{Light and Matter} Chapter 2

\begin{questions}

\question[1]
In graphical terms, an object's velocity is interpreted as what part of a position versus time graph?
\fillwithlines{0.5in}

\question[2]
In graphical terms, a positive slope characterizes what kind of line? What does a negative slope characterize?
\fillwithlines{0.5in}

\question[1]
For a position versus time graph with a curvy line, the velocity at any given moment is found using what? (Hint: look at figures and their captions)
\fillwithlines{0.75in}

\question[2]
The standard convention for graphing has \textit{x} on which axis (horizontal or upright)? What about \textit{t} (horizontal or upright)?
\fillwithlines{0.5in}

\end{questions}

\noindent Questions from \textit{College Physics} Chapter 2

\begin{questions}
\question[1]
The slope of a position versus time graph is determined how? (See equation 2.92)
\fillwithlines{0.5in}

\question[1]
If an object is accelerating at a constant rate ($a \neq 0$) is the position versus time graph straight or curvy?
\fillwithlines{0.5in}

\question[1]
The slope of a velocity versus time graph represents what physical quantity (displacement, velocity, or acceleration)?
\fillwithlines{0.5in}

\question[1] 
If an object's acceleration is changing, is the velocity versus time graph straight or curvy?
\fillwithlines{0.5in}

\end{questions}


\end{document}