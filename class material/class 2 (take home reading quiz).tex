\documentclass[addpoints,12pt]{exam}
%\documentclass[12pt]{article}
\usepackage[letterpaper, margin=0.75in]{geometry}
\usepackage{graphicx}
\usepackage{enumitem}
\usepackage{booktabs}

\begin{document}
\footer{}{Page \thepage\ of \numpages}{}

\begin{flushright}
\makebox[0.5\textwidth]{\large Name:\enspace\hrulefill}
\vspace{0.2in}

\makebox[0.5\textwidth]{\large Date:\enspace\hrulefill}
\end{flushright}

\begin{center}
\includegraphics[width=10cm]{../images/logo.png}
\end{center}

\begin{center}
\noindent{\LARGE Conceptual Physics \\ Reading Quiz 2}
\end{center}

\noindent\begin{large}\textbf{Due Date: Feb. 16 (before class)}\end{large}
\vspace{0.2in}

This reading assignment covers the material that we will discuss and work on next week. Our class activities will \textbf{assume} that you have read the assigned material, therefore it is \textbf{very important} that you do so to get the most out of the class!
\begin{enumerate}
	\item \textit{Light and Matter}, Chapter 2 (Sections 1, 2 and 5)
	
	Section 1 introduces a useful concept, a \textit{center of mass}. This is a useful concept in physics, and for the rest of the semester when we discuss an object's motion we will primarily be referring to the motion of an object's center of mass. (You don't need to pay attention to the technical difference between it and the geometrical center, or how jumping while flailing your limbs changes your motion). In section 2, the author introduces new notation (new variables and meanings). It's OK if you don't follow everything in the text regarding ``things" and ``changes in things"- we will go through it in class. Unless you want to jump ahead, there is no need to read section 3 and 4 just yet. Section 5 introduces the idea of adding velocities to describe relative motion, and the importance of negative numbers, which will be very important to us.
	
	\item \textit{College Physics}, Chapter 2 (Sections 1 to 5)

	This chapter introduces a lot of the vocabulary and formalism we will be using when solving problems related to motion. It's OK if you don't follow what all the new terms mean, we will discuss them in class. In section 2, the author goes into detail about the difference between ``scalars" and ``vectors", and then illustrates them in detail by comparing velocity, speed, distance, displacement, etc. in section 3. We will go over all these differences in class, however if you want to prepare more I would suggest coming up with a list of terms from the text, and indicting whether they are classified as scalar or vector (and see how they are related). The text also talks about the difference between \textit{average} and \textit{instantaneous} velocity/acceleration. Don't worry too much about this difference yet, we will talk about it more next week when we discuss graphs.
	Figure 2.11, 2.17 and 2.21 also introduce graphs, which we will be discussing next week, so don't worry too much about them. I recommend focusing on the worked examples that don't use graphs.
	Section 2.5 uses a lot of equations (in particular 2.35, 2.40 and 2.46) that form the basis of the kinematic equations we will use in class next week. Please make sure you are somewhat familiar with these three equations.
	
	\item OPTIONAL: \textit{Light and Matter}, Chapter 3 (Sections 1 to 4, and 6)
	
	A key component in this chapter is relating acceleration to velocity. Please focus mostly on section 6, which talks about the actual kinematic equations. In all the section, the author finds mathematical relationships between quantities using graphs. You don't need to pay attention to the details of the graphs - we will discuss them in detail next week.
	
\end{enumerate}

As part of the reading, please complete the pre-class quiz \textit{before} coming to class. They will be collected at the very beginning.
 
\clearpage

\begin{flushright}
Score: \hspace{0.2in} / \numpoints ~ points
\end{flushright}

\noindent Questions from \textit{Light and Matter} Chapter 2.

\begin{questions}

\question[1]
Ever object has a balance point. What is this point referred to in physics?
\fillwithlines{0.5in}

\question[1]
What does the Greek letter $\Delta$ (delta) mean?
\fillwithlines{0.5in}

\question[1]
When defining a position variable, you need to decide where to put $x=0$ and define a positive direction. What are you choosing when you do this?
\fillwithlines{0.5in}

\question[1]
Adding velocities has the significance of comparing what kind of motion?
\fillwithlines{0.5in}
\end{questions}

\noindent Questions from \textit{College Physics} Chapter 2.

\begin{questions}
\question[1]
What is the definition of \textit{kinematics}?
\fillwithlines{0.5in}

\question[1]
What is the definition of \textit{displacement}? (Feel free to use words or an equation)
\fillwithlines{0.5in}

\question[1]
What is an \textit{average velocity}? (Feel free to use words or an equation)
\fillwithlines{0.5in}

\question[1]
What is an \textit{average acceleration}? (Feel free to use words or an equation)
\fillwithlines{0.5in}

\question[2]
Please write down equations 2.35, 2.40 and 2.46.
\fillwithlines{0.75in}

\end{questions}

\clearpage

\noindent Optional Questions from \textit{Light and Matter} Chapter 3.

\begin{questions}
\bonusquestion[1]
What are the main variables related to the value of \textit{g} on Earth?
\fillwithlines{0.5in}

\end{questions}



\end{document}