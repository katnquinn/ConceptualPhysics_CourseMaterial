\documentclass[addpoints,12pt]{exam}
%\documentclass[12pt]{article}
\usepackage[letterpaper, margin=0.75in]{geometry}
\usepackage{graphicx}
\usepackage{enumitem}
\usepackage{booktabs}
\usepackage{tabularx}
\usepackage{color}
\usepackage{wrapfig}

\begin{document}
\footer{}{Page \thepage\ of \numpages}{}

\begin{flushright}
\makebox[0.5\textwidth]{\large Name:\enspace\hrulefill}
\vspace{0.2in}

\end{flushright}

\begin{center}
\includegraphics[width=10cm]{../images/logo.png}
\end{center}

\begin{center}
\noindent{\LARGE Conceptual Physics \\ Second Test\\ April 25, 2018 \\}
\end{center}

\vspace{0.5in}

\begin{large}
You are free to use all notes on your two-sided cheat sheet. There are extra blank sheets at the end, which can be used for calculations, and if you require more please ask and be sure to include them when you hand back the test. Please be sure to include all your work and calculations.

There are \numquestions~problems for a total of \numpoints~points. (One of the questions is a bonus though.)
\end{large}
\vspace{0.2in}


 
\clearpage

\begin{flushright}
Score: \hspace{0.2in} / \numpoints ~ points
\end{flushright}

\begin{questions}
\question \textbf{Drawing Field Lines:} Four point charges are arranged in 2 different configurations, resulting in different electric fields. Assume that the size of all charges are the same, and consider only the fact that some are positive and others are negative.

\begin{parts}

\part[2] Draw the electric field lines around the following:
\vspace{0.25in}
\begin{center}
\includegraphics[width=3in]{../images/4chargesA.png}
\end{center}
\vspace{0.25in}

\part[2] Draw the electric field lines around the following:
\vspace{0.25in}
\begin{center}
\includegraphics[width=3in]{../images/4chargesB.png}
\end{center}
\vspace{0.25in}
\end{parts}

\clearpage
\question \textbf{Lightening on a Train:} A train is moving past a platform at a velocity $v$. Lightning bolts strike the front and back of the train, scorching both the train and the platform, as the train passes the platform. An observer at rest on the platform says the strikes were simultaneous. (Hint: For this problem, it may be easiest to draw diagrams on the extra scratch paper.)
\begin{center}
	\includegraphics[width=0.7\textwidth]{../images/test2_lightning.png}
	\end{center}
	
	\begin{parts}
		\part[2] A person on the platform says that the distance between \textit{scorched marks on the platform} is \textit{D}. To a person on the train, the distance between \textit{scorch marks on the platform} is:
		\begin{choices}
			\choice Longer than \textit{D}.
			\choice Shorter than \textit{D}.
			\choice Exactly \textit{D}.
			\choice Cannot say, we need more information.
		\end{choices}
		\part[2] A person in the train measures the length of the train to be \textit{L}. To a person on the platform, is the train:
		\begin{choices}
			\choice Longer than \textit{L}.
			\choice Shorter than \textit{L}.
			\choice Exactly \textit{L}.
			\choice Cannot say, we need more information.
		\end{choices}
		\part[2] A person on the platform say that the distance between the \textit{scorch marks on the train} is also \textit{D} (since they see the strikes as happening at the same time). To a person on the train, the distance between \textit{scorch marks on the train} is:
		\begin{choices}
			\choice Longer than \textit{D}.
			\choice Shorter than \textit{D}.
			\choice Exactly \textit{D}.
			\choice Cannot say, we need more information.
		\end{choices}
		\part[2] A person on the platform says the two strikes were simultaneous. What would a person on the train say?
		\begin{choices}
			\choice They would agree, the strikes happened simultaneously.
			\choice They would disagree, and say that the strike at \textit{A'} occurred before the strike at \textit{B'}
			\choice They would disagree, and say that the strike at \textit{B'} occurred before the strike at \textit{A'}
			\choice Cannot say, it is impossible for the person on the train to determine the order the strikes occurred.
		\end{choices}
	\end{parts}
	
	
\clearpage
\question \textbf{Electron Energies:} Three electrons have different energies. Using the ideas of wave/particle duality, we can represent the electrons as waves. The following images shows what their wavelength looks like:

\begin{center}
\includegraphics[width=4in]{../images/deBroglie.png}
\end{center}

The following questions will ask about how their energies, $E_I$, $E_{II}$, and $E_{III}$, relate.

\begin{parts}
	\part[2] How do $E_I$ and $E_{II}$ compare?
		\begin{choices}
			\choice $E_{I} = E_{II}$
			\choice $E_{I} > E_{II}$
			\choice $E_{I} < E_{II}$
		\end{choices}
	\part[2] How do $E_I$ and $E_{III}$ compare?
		\begin{choices}
			\choice $E_{I} = E_{III}$
			\choice $E_{I} > E_{III}$
			\choice $E_{I} < E_{III}$
		\end{choices}
	\part[2] How do $E_{II}$ and $E_{III}$ compare?
		\begin{choices}
			\choice $E_{II} = E_{III}$
			\choice $E_{II} > E_{III}$
			\choice $E_{II} < E_{III}$
		\end{choices}
\end{parts}

	\clearpage
	
	\question \textbf{Balls in a Jar:} You have a jar with 6 red balls and 4 green balls.
\begin{parts}
\part[2] What is the probability of randomly drawing a red ball? What is the probability of randomly drawing a green ball?
\vspace{1.5in}
\part[2] You randomly draw one ball, put it back, and draw a ball again. What is the probability of drawing a red and a green ball, in no particular order?
\vspace{1.5in}
\part[2] You randomly draw one ball then draw another without putting back the first. What is the probability of drawing two red balls?
\vspace{1.5in}
\part[2] You randomly draw one ball then draw another without putting back the first. What is the probability of drawing a red and a green ball, in no particular order?
\vspace{1.5in}
\end{parts}

\clearpage
\question \textbf{Photoelectric Effect:} Consider the following experimental setups. Light of a given wavelength is incident on one of two  metal plates sealed inside a vacuum tube. The two plates are connected in circuit to an ammeter, a device that measures electric current. The metal plates have been thoroughly cleaned so that when blue light ($\lambda = 450$~nm) is incident on the lower plate, electrons are ejected.
\begin{parts}
\part[2] There are two lightbulbs, which emit light at 300 nm and 400 nm. These wavelengths are shorter than blue light. Is the energy of the photons emitted by these two lightbulbs:
	\begin{choices}
		\choice The same as the energy contained in blue light (with wavelength 450 nm).
		\choice Greater than the energy contained in a blue light (with wavelength 450 nm).
		\choice Less than the energy contained in a blue light (with wavelength 450 nm).
	\end{choices}
\part[2] In the following two setups, light of 300 nm and 400 nm wavelength (ultraviolet light) is incident on the lower plates. The number of photons landing on the metal plate per second is the same in both setups (they have the same intensity). Compare the number and energy of emitted electrons in the two setups.
\begin{center}
\includegraphics[width=0.45\textwidth]{../images/test2_300nm.png} \includegraphics[width=0.45\textwidth]{../images/test2_400nm.png}
\end{center}
	\begin{choices}
		\choice The two setups emit the same number of electrons, however the setup on the left emits electrons with greater energies.
		\choice The two setups emit the same number of electron, however the setup on the right emits electrons with greater energies.
		\choice The two setups emit electrons at the same energies, however the setup on the left emits more.
		\choice The two setups emit electrons at the same energies, however the setup on the right emits more.
		\choice The two setups emit the same number of electrons and at the same energies.
	\end{choices}
\vspace{2in}
\clearpage
\part[2] In the following two setups, light of 300 nm wavelength (ultraviolet light) is incident on the lower plates. There are three times as many photons landing on the metal plate per second in the setup on the left (the left light has a greater intensity). Compare the number and energy of emitted electrons in the two setups.
\begin{center}
\includegraphics[width=0.45\textwidth]{../images/test2_300nm.png} \includegraphics[width=0.45\textwidth]{../images/test2_300nm1.png}
\end{center}
	\begin{choices}
		\choice The two setups emit the same number of electrons, however the setup on the left emits electrons with greater energies.
		\choice The two setups emit the same number of electron, however the setup on the right emits electrons with greater energies.
		\choice The two setups emit electrons at the same energies, however the setup on the left emits more.
		\choice The two setups emit electrons at the same energies, however the setup on the right emits more.
		\choice The two setups emit the same number of electrons and at the same energies.
	\end{choices}

\end{parts}
	
\clearpage
\bonusquestion \textbf{Creating Elements:} A nuclear physicist in a lab wishes to create oxygen. The graph below shows the binding energies for different elements and isotopes:
\noindent\begin{center}
\includegraphics[width=0.9\textwidth]{../images/bindingEnergies.png}
\end{center}
\begin{parts}
	\part[2] She combines $^3He$ to create $^{16}O$. Overall, would this release or require energy?
		\vspace{0.5in}
	\part[2] She breaks apart $^{56}Fe$ to create $^{16}O$. Overall, would this release or require energy?
		\vspace{0.5in}
\end{parts}
\end{questions}

\clearpage
This page is left blank for calculations.

\clearpage
This page is left blank for calculations.

\clearpage
This page is left blank for calculations.


\end{document}